\section{Introduction}

For some random field \(Y\) assume that labels \(Y=Y(X)\) can be generated
from locations \(X\). To fit a supervised machine learning model to this data
\((X,Y)\), one would parametrize a model \(f_\theta\) and measure the distance
between \(f_\theta(X)\) and \(Y\) using a loss function,
e.g.
\[
	l(\theta, X,Y) = \|f_\theta(X)-Y\|^2.
\]
Then we would want to find a minimum of the random field
\[
	\rf(\theta) := \E[l(\theta, X,Y(X))\mid Y].
\]
In \fxnote{self-cite} random field informed gradient methods are discussed. For
these methods to work, they need information about the covariance structure of
\(\rf\). In this work we want to estimate this covariance structure.

More generally, for a random field \(Z:\real^\dimension \to \real\) with high
input dimension \(\dimension\), the covariance function is defined as
\[
	\C(x,y) = \cov(Z(x), Z(y)).
\]
If the covariance structure depends only on the distance of the points
\[
	\C(x,y) = \C(\|x-y\|),
\]
it is called isotropic. Isotropy is already a common assumption in geostastics
\parencite[e.g.][Chapter 2]{mullerCollectingSpatialData2007} because it greatly
reduces the amount of data needed for estimation, but for high-dimensional
random fields, the number of directions is likely higher than the number of
points we can afford to sample, so some form of rotation invariance is vital to
make this problem tractable.

The typical generalization beyond isotropy allows linear transformations of the
input space \parencite[e.g.][Section 5.1]{williamsGaussianProcessesMachine2006}
or \parencite{sampsonNonparametricEstimationNonstationary1992}, i.e.
\[
	\C(x,y) = \C((x-y)^T M (x-y))
\]
where \(M\) is a symmetric matrix such that it can be represented by \(LL^T\)
which implies
\[
	\C(x,y) = \C(\|L^T(x-y)\|)
\]
As \(M\) is a \(\dimension\times\dimension\) matrix, it is then typically
assumed to fulfill
\[
	M = \diag(\sigma_1,\dots,\sigma_\dimension) + \Gamma\Gamma^T
\]
where \(\Gamma\) is merely a \(k\times\dimension\) matrix for \(k\ll\dimension\).
But even then this still implies a number of parameters in the order of
\(O(\dimension)\) we need to estimate. This is much too big in our case, so we
are not going to bother. With appropriate sparsity assumptions this might
somehow be salvageable, but this is beyond the scope of this work.

The isotropy assumption implicitly assumes the random field is stationary,
i.e. invariant to shifts. A generalization one can make (we will make)
is, that only the increments are stationary (isotropic). More precisely, the
random field \(Z\) is \emph{intrinsically stationary}, if the increments
\[
	Z_h(x) := Z(x+h) - Z(x)
\]
are stationary for every \(h\in\real^\dimension\). Therefore
\[
	\C_h(x-y) = \C_h(x,y) = \cov(Z_h(x), Z_h(y))
\]
and we define the centered variogram
\[
	2\gamma(h):= \C_h(0) = \cov(Z_h(x), Z_h(x))= \var(Z_h(x)).
\]
For centered random fields (\(\E[Z]=0\)), the centered variogram is identical
to the uncentered variogram
\[
	\gamma(h) = \tfrac12\E[(Z(x+h)-Z(x))^2].
\]
Assuming \(Z\) was stationary to begin with, the binomial formula yields 
\[
	\C(x-y) = \C(0) - \gamma(x-y).
\]
In parametric models, estimation of the variogram is therefore often equivalent
to estimating the covariance function.
