\section{Experiement Design}

To estimate the covariance structure, we need evaluations
\(\rf(x_1),\dots,\rf(x_n)\) of our random
field \(\rf\). But where should we sample our Random Field? I.e. what should
\(x_1,\dots, x_n\) be? A grid or otherwise regular lattice as is common in
geostatistical experiment design \parencite[e.g.][Section 4.1]{mullerCollectingSpatialData2007} does not work, because even for a trivial
grid with only two options in every direction results in \(n=2^\dims\) points.
This is way too much for high dimensions \(d\) and offers little variety in
the distance of points \(x_i\).

\subsection{Random Evaluation Points}

So instead, we are going to select the evaluation points randomly. Let the point
\(X_i\) be a multivariate normal \(\normal(0,\frac{\sigma_i^2}\dims
\identity_{\dims\times \dims})\) vector describing the \(i\)-th point
in the evaluation space \(\real^\dims\). Notice how the strong law of large
numbers causes our points \(X_i\) to all be of similar length in high dimension
\[
	\|X_i\|^2 = \sum_{k=1}^\dims (X_i^{(k)})^2
	= \frac{\sigma_i^2}{\dims} \sum_{k=1}^\dims (Y_i^{(k)})^2 
	\overset{\text{SLLN}}\longrightarrow\sigma_i^2\underbrace{\E[(Y_i^{(1)})^2]}_{=1} \quad (\dims\to \infty)
\]
for \(X_i^{(k)} = \frac{\sigma_i}{\sqrt{\dims}}Y_i^{(k)}\), where \(Y_i^{(k)}\) is
independently \(\normal(0,1)\) distributed. If we had not scaled our random
variables by \(1/{\sqrt{\dims}}\), their average length would increase
with dimension, but their length would still become more similar to
\(\sigma_i\sqrt{\dims}\) relative to their total length. In other words:
a fixed number of significant digits in floating point notation would still get
closer and closer in higher dimension.

The takeaway is, that independent identically distributed random vectors are
all roughly of the same length in high dimension. And the distance between
two independent points is just a random vector itself
\[
	X_i-X_j\sim\normal\left(
		0,\frac{\sigma_i^2+\sigma_j^2}\dims\identity_{\dims\times\dims}
	\right),
\]
and therefore\footnote{
	These observations are directly related to the fact, that in high dimension
	random vectors are roughly orthogonal, as they fulfill the pythagorean
	theorem in the limit
	\[
		2\langle X_i, X_j\rangle = \|X_i-X_j\|^2 - (\|X_i\|^2 + \|X_j\|^2) \to 0.
	\]
}
\[
	\|X_i- X_j\|^2 \to \sigma_i^2 + \sigma_j^2
	\quad\overset{\text{cont. map}}\implies \quad
	\|X_i-X_j\|\to\sqrt{\sigma_i^2+\sigma_j^2}.
\]
So it is crucial, that the \(X_i\) are not identically distributed. Otherwise
all our points would have the same distance \(\sqrt{2}\sigma\) from each other. But
in order to estimate an (isotropic) covariance function,
we need measurements at \emph{varying} distances from each other. Otherwise we would
just get a covariance estimate for one particular distance. It is also noteworthy
that we only really used the expectation and variance of the entries of \(X_i\).
The normal distribution is not crucial here.

\subsection{Selecting the Variances \texorpdfstring{\((\sigma_1^2,\dots,\sigma_n^2)\)}{(σ₁²,...,σₙ²)}}

As we have already established, that we do not want the distribution of
variances \(\sigma_i^2\) to be a dirac measure, the question is: What is a
useful distribution for \(\sigma_i^2\)? If we have
\(\sigma_i^2\overset{\text{iid}}\sim\mu\), then the empirical measure
\[
	\mu_n = \frac1n\sum_{k=1}^n \delta_{\sigma_i^2}
\]
for dirac measures \(\delta_x\), approximate \(\mu\) for \(n\to\infty\). Picking
a random pair \((\sigma_i^2,\sigma_j^2)\) is equivalent from sampling from the
product measure \(\mu_n \times \mu_n\) which approximates \(\mu\times\mu\).
The distribution of distances
\[
	\|X_i-X_j\| \approx \sqrt{\sigma_i^2 + \sigma_j^2}
\]
is therefore approximately the distribution of \(\sqrt{X+Y}\) where
\(X,Y\overset{\text{iid}}\sim\mu\). If we had
\(\sigma_i^2\overset{\text{iid}}{\sim}\Gamma(\tfrac14,2\theta)\) for example,
then we have
\[
	\sigma_i^2 + \sigma_j^2 \sim \Gamma(\tfrac12,2\theta) = \theta\chi^2(1),
\]
which in turn implies that the distances \(\sqrt{\sigma_i^2 + \sigma_j^2}\)
are half-normal distributed.

It appears to be common knowledge in the statistical community, ``that the
behavior of [the variogram] \(\gamma(h)\) is most critical for \(|h|\) small;
hence [the number of samples at distance \(h\)] \(n(h)\) should be large for
\(|h|\) small'' \parencite{warrickOptimizationSamplingLocations1987}. A similar
statement was made by Martin Schlather in personal communication. But there
does not appear to be any theory, to substantiate this empirical knowledge.
Nevertheless the half-normal distribution fulfills the requirement for many
small distances, so we are happy it is so easy to obtain.

\begin{algorithm}
	\caption{Select \(n\) Sample Points}	
	Sample \(\sigma_1,\dots,\sigma_n\) iid from \(\Gamma(1,\theta)\)\tcp{Take
	note of the unfortunate fact, that we have the hyperparameter \(\theta\)
	left-over here}
	\For{\(i\) in \(1,\dots, n\)}{
		Sample \(X_i\) independently from \(\normal(0, \frac{\sigma_i^2}\dims
		\identity_{\dims\times\dims})\)
	}
	\Return{\((X_1,\dots,X_n)\)}
\end{algorithm}

